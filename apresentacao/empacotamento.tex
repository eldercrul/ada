\documentclass{beamer}
\usepackage[brazil]{babel}
\usepackage[utf8]{inputenc}
%\usepackage{verbatim}

\title{Empacotamento e Tipagem Genérica em ADA}
\author{Aramis Fernandes \& Elder Crul}

\begin{document}

    % Uma das formas possiveis para declarar um frame:
    \frame{\titlepage}

    % [2]
    \frame
    {
        \frametitle{Sumário}
        \tableofcontents
    }

    % Manda os highlight da secao corrente:
    \AtBeginSection[]
    {
        \begin{frame}{Sumário}
            \tableofcontents[currentsection]
        \end{frame}
    }

    % Aqui a secao comeca efetivamente:
    \section{Empacotamento}

    % Cada frame eh um slide:
    \begin{frame}{O que é?}
        Segundo o Prof. Direne, é a Arte de ocultar gradualmente a informação.
    \end{frame}

    %%%
    \begin{frame}{E em ADA...}
        \begin{itemize}
            \item<1-> Existem dois tipos de pacotes:
            \begin{itemize}
                \item ADT (\textit{Abstract Data Type}): visíveis e privados;
                \item ASM (\textit{Abstract State Machine}): objeto.
            \end{itemize}
            \item<2-> ASM.public: operações e não atributos.
            \item<2-> ASM.private: *TODOS* os atributos.
        \end{itemize}
    \end{frame}

    %%% Exemplos de código:
    \begin{frame}[containsverbatim]
    
        \frametitle{Exemplo:}
\begin{verbatim}
-- Definicao de um pacote onde especificacoes,
-- variaveis, tipos, constantes, etc.,
-- sao todos visiveis pelos programas que
-- usarao o pacote:

package Public_Only_Package is

    type Range_10 is range 1 .. 10;

end Public_Only_Package;
\end{verbatim}
    \end{frame}

    \begin{frame}[containsverbatim]
    
        \frametitle{Exemplo:}
\begin{verbatim}
-- Pacote privado:
package Not_That_Public_Package is

    type Private_Type is private;

private

   type Private_Type is array (1 .. 10) of Integer;

end Not_That_Public_Package;
\end{verbatim}
    \end{frame}

    % Manda os highlight da secao corrente:
    \AtBeginSection[]
    {
        \begin{frame}{Sumário}
            \tableofcontents[currentsection]
        \end{frame}
    }
    % A outra secao:
    \section{Tipos Genéricos}
    \begin{frame}{Tipagem Genérica}
        \begin{itemize}
            \item<1-> 1;
            \item<2-> 2;
        \end{itemize}
    \end{frame}

\end{document}
