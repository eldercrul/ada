\documentclass{beamer}
\usepackage[brazil]{babel}
\usepackage[utf8]{inputenc}

\title{Empacotamento e Tipagem Genérica em ADA}
\author{Aramis Fernandes \& Elder Crul}

\begin{document}

    % Uma das formas possiveis para declarar um frame:
    \frame{\titlepage}

    % [2]
    \frame
    {
        \frametitle{Sumário}
        \tableofcontents
    }

    % Manda os highlight da secao corrente:
    \AtBeginSection[]
    {
        \begin{frame}{Sumário}
            \tableofcontents[currentsection]
        \end{frame}
    }

    % Aqui a secao comeca efetivamente:
    \section{Empacotamento}

    % Cada frame eh um slide:
    \begin{frame}{O que é?}
        \begin{itemize}
            \item<1-> Modularização;
            \item<2-> Portabilidade;
            \item<3-> \textit{Et cetera}.
        \end{itemize}
    \end{frame}

    %%%
    \begin{frame}{Mas em ADA...}
        \begin{itemize}
            \item<1-|alert@1>Nomes de pacotes e arquivos são relacionados,
            prevenindo colisões.
            \item<2-> Um arquivo com definições (.h), e um arquivo com o bloco de
            execução (.c).
            \item<3-> Módulos podem ser compilados separadamente.
        \end{itemize}
    \end{frame}

    %%% Exemplos de código:
    \begin{frame}[fragile]
        \frametitle{Exemplo:}
         package Public\_Only\_Package is
             type Range\_10 is range 1 .. 10;
             end Public\_Only\_Package;
    \end{frame}


    % Manda os highlight da secao corrente:
    \AtBeginSection[]
    {
        \begin{frame}{Sumário}
            \tableofcontents[currentsection]
        \end{frame}
    }
    % A outra secao:
    \section{Tipos Genéricos}
    \begin{frame}{Tipagem Genérica}
        \begin{itemize}
            \item<1-> 1;
            \item<2-> 2;
        \end{itemize}
    \end{frame}

\end{document}
